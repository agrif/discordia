\chapter*{Introduction}
\vspace{4em} % FIXME make sure this is set so the pagebreak below is nice

You hold in your hands one of the Great~Books of our century\fnord{ fnord}.

Some Great~Books are recognized at once with a fusillade of critical
huzzahs and gonfolons, like Joyce's \textit{Ulysses}. Others appear
almost furtively and are only discovered 50 years later, like
\textit{Moby~Dick} or Mendel's great essay on genetics. The
\textit{Principia~Discordia} entered our space-time continuum almost
as unobtrusively as a cat-burglar\index{cat-burglar} creeping over a
windowsill.

In 1968, virtually nobody had heard of this wonderful book. In 1970,
hundreds of people coast to coast were talking about it and asking the
identity of the mysterious author,
Malaclypse~the~Younger\index{Mal-2}. Rumors swept across the
continent, from New~York to Los~Angeles, from Seattle to
St.~Joe. Malaclypse was actually Alan~Watts, one heard. No, said
another legend -- the \textit{Principia} was actually the work of the
Sufi Order. A third, very intriguing myth held that Malaclypse was a
pen-name for Richard~M.~Nixon\index{Nixon, Richard M.}, who had
allegedly composed the \textit{Principia} during a few moments of
lucidity. I enjoyed each of these yarns and did my part to help spread
them. I was also careful never to contradict the occasional rumors
that I had actually written the whole thing myself during an acid
trip.

The legendry, the mystery, the cult grew slowly. By the mid-1970's,
thousands of people, some as far off as Hong Kong and Australia, were
talking about the \textit{Principia}, and since the original was out
of print by then, xerox copies were beginning to circulate here and
there.

When the \textit{Illuminatus}~trilogy appeared in 1975, my co-author,
Bob~Shea, and I both received hundreds of letters from people
intrigued by the quotes from the \textit{Principia} with which we had
decorated the heads of several chapters. Many, who had already heard
of the \textit{Principia} or seen copies, asked if Shea and I had
written it, or if we had copies available. Others wrote to ask if it
were real, or just something we had invented the way H.P.~Lovecraft
invented the \textit{Necronomicon}. We answered according to our
moods, sometimes telling the truth, sometimes spreading the most
Godawful lies and myths we could devise\fnord{ fnord}.

Why not? We felt that this book was a true Classic
(\textit{literatus~immortalis}) and, since the alleged intelligentsia
had not yet discovered it, the best way to keep its legend alive was
to encourage the mythology and the controversy about it. Increasingly,
people wrote to ask me if Timothy Leary had written it, and I almost
always told them he had, except on Fridays when I am more whimsical,
in which case I told them it had been transmitted by a canine
intelligence -- vast, cool, and unsympathic -- from the Dog Star,
Sirius\index{Dog Star}.

Now, at last, the truth can be told.

\pagebreak

Actually, the \textit{Principia} is the work of a time-travelling
anthropologist from the 23rd Century. He is currently passing among us
as a computer specialist, bon~vivant and philosopher named
Gregory~Hill. He has also translated several volumes of Etruscan
erotic poetry\index{erotic poetry!Etruscan}, under another pen-name,
and in the 18th Century was the mysterious Man in Black who gave
Jefferson the design for the Great Seal of the United States.

I have it on good authority that he is one of the most accomplished
time-travelers in the galaxy and has visited Earth many times in the
past, using such cover-identities as Zeno~of~Elias,
Emperor~Norton\index{Emperor Norton}, Count~Cagliostro,
Guilliame~of~Aquaitaine, etc. Whenever I question him about this, he
grows very evasive and attempts to persuade me that he is actually
just another 20th Century Earthman and that all my ideas about his
extraterrestrial and extratemporal origin are delusions. Hah! I am not
that easily deceived. After all, a time-travelling anthropologist
would say just that, so that he could observe us without his presense
causing cultureshock.

I understand that he has consented to write an Afterword to this
edition. He'll probably contradict everything I've told you, but don't
believe a word he says\fnord{ fnord}. He is a master of the deadpan
put-on, the plausible satire, the philosophical leg-pull and all the
branches of guerilla ontology.

For full benefit to the Head, this book should be read in conjunction
with \textit{The~Illuminoids} by Neal~Wilgus (Sun~Press,
Albuquerque,~NM) and \textit{Zen~Without~Zen~Masters} by
Camden~Benares (And/Or~Press, Berkeley, California). ``We~are
operating on many levels here'', as Ken~Kesey used to say.

In conclusion, there is no conclusion. Things go on as they always
have, getting weirder all the time.

Hail Eris. All hail Discordia.\fnord{ Fnord?}

\begin{blockattribution}
Robert Anton Wilson \\
\textit{International Arms and Hashish Inc.} \\
Darra Bazar, Kohat
\end{blockattribution}
